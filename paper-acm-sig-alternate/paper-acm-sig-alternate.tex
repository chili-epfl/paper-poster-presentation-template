% This is "sig-alternate.tex" V2.1 April 2013
% This file should be compiled with V2.5 of "sig-alternate.cls" May 2012
%
% This example file demonstrates the use of the 'sig-alternate.cls'
% V2.5 LaTeX2e document class file. It is for those submitting
% articles to ACM Conference Proceedings WHO DO NOT WISH TO
% STRICTLY ADHERE TO THE SIGS (PUBS-BOARD-ENDORSED) STYLE.
% The 'sig-alternate.cls' file will produce a similar-looking,
% albeit, 'tighter' paper resulting in, invariably, fewer pages.
%
% ----------------------------------------------------------------------------------------------------------------
% This .tex file (and associated .cls V2.5) produces:
%       1) The Permission Statement
%       2) The Conference (location) Info information
%       3) The Copyright Line with ACM data
%       4) NO page numbers
%
% as against the acm_proc_article-sp.cls file which
% DOES NOT produce 1) thru' 3) above.
%
% Using 'sig-alternate.cls' you have control, however, from within
% the source .tex file, over both the CopyrightYear
% (defaulted to 200X) and the ACM Copyright Data
% (defaulted to X-XXXXX-XX-X/XX/XX).
% e.g.
% \CopyrightYear{2007} will cause 2007 to appear in the copyright line.
% \crdata{0-12345-67-8/90/12} will cause 0-12345-67-8/90/12 to appear in the copyright line.
%
% ---------------------------------------------------------------------------------------------------------------
% This .tex source is an example which *does* use
% the .bib file (from which the .bbl file % is produced).
% REMEMBER HOWEVER: After having produced the .bbl file,
% and prior to final submission, you *NEED* to 'insert'
% your .bbl file into your source .tex file so as to provide
% ONE 'self-contained' source file.
%
% ================= IF YOU HAVE QUESTIONS =======================
% Questions regarding the SIGS styles, SIGS policies and
% procedures, Conferences etc. should be sent to
% Adrienne Griscti (griscti@acm.org)
%
% Technical questions _only_ to
% Gerald Murray (murray@hq.acm.org)
% ===============================================================
%
% For tracking purposes - this is V2.0 - May 2012

\documentclass{sig-alternate-05-2015}

\usepackage{booktabs}
\usepackage{graphicx}
\usepackage[T1]{fontenc}
\usepackage[pdflang={en-US},pdftex,bookmarks=true]{hyperref}
\usepackage[inline]{enumitem}
\usepackage{microtype}
\usepackage[utf8]{inputenc}
\usepackage{xspace}
\usepackage[caption=false,font=small]{subfig}
\usepackage{amsmath}

\graphicspath{{../figures/}}

\newcommand{\eg}{\textit{e.g.}\xspace}
\newcommand{\etal}{\textit{et al.}\xspace}
\newcommand{\ie}{\textit{i.e.}\xspace}
\newcommand{\etc}{\textit{etc.}\xspace}
\newcommand{\vs}{\textit{vs.}\xspace}

\begin{document}

% Copyright
\setcopyright{acmcopyright}
%\setcopyright{acmlicensed}
%\setcopyright{rightsretained}
%\setcopyright{usgov}
%\setcopyright{usgovmixed}
%\setcopyright{cagov}
%\setcopyright{cagovmixed}

\doi{XXXXX/XXXX}
\isbn{XXXXXXXXXX}
\conferenceinfo{CONF 'YY}{Month DD--DD, YYYY, City, Country}

\acmPrice{\$XX.XX}

%
% --- Author Metadata here ---
\conferenceinfo{CONF}{'YY City, Country}
%\CopyrightYear{2007} % Allows default copyright year (20XX) to be over-ridden - IF NEED BE.
%\crdata{0-12345-67-8/90/01}  % Allows default copyright data (0-89791-88-6/97/05) to be over-ridden - IF NEED BE.
% --- End of Author Metadata ---

\title{Alternate {\ttlit ACM} SIG Proceedings Paper
%\titlenote{(Produces the permission block, and copyright information). For use with SIG-ALTERNATE.CLS. Supported by ACM.}
}
%\subtitle{[Extended Abstract]
%\titlenote{A full version of this paper is available as \textit{Author's Guide to Preparing ACM SIG Proceedings Using \LaTeX$2_\epsilon$\ and BibTeX} at \texttt{www.acm.org/eaddress.htm}}}

% You need the command \numberofauthors to handle the 'placement
% and alignment' of the authors beneath the title.
%
% For aesthetic reasons, we recommend 'three authors at a time'
% i.e. three 'name/affiliation blocks' be placed beneath the title.
%
% NOTE: You are NOT restricted in how many 'rows' of
% "name/affiliations" may appear. We just ask that you restrict
% the number of 'columns' to three.
%
% Because of the available 'opening page real-estate'
% we ask you to refrain from putting more than six authors
% (two rows with three columns) beneath the article title.
% More than six makes the first-page appear very cluttered indeed.
%
% Use the \alignauthor commands to handle the names
% and affiliations for an 'aesthetic maximum' of six authors.
% Add names, affiliations, addresses for
% the seventh etc. author(s) as the argument for the
% \additionalauthors command.
% These 'additional authors' will be output/set for you
% without further effort on your part as the last section in
% the body of your article BEFORE References or any Appendices.

\numberofauthors{8} %  in this sample file, there are a *total*
% of EIGHT authors. SIX appear on the 'first-page' (for formatting
% reasons) and the remaining two appear in the \additionalauthors section.
%
\author{
% You can go ahead and credit any number of authors here,
% e.g. one 'row of three' or two rows (consisting of one row of three
% and a second row of one, two or three).
\alignauthor Author Author\\ %\titlenote{Some note on the bottom of the page.}
       \affaddr{Some Institute}\\
       \affaddr{Street Address}\\
       \affaddr{City, Country}\\
       \email{email@email.com}
\alignauthor Author Author\\ %\titlenote{Some note on the bottom of the page.}
       \affaddr{Some Institute}\\
       \affaddr{Street Address}\\
       \affaddr{City, Country}\\
       \email{email@email.com}
\alignauthor Author Author\\ %\titlenote{Some note on the bottom of the page.}
       \affaddr{Some Institute}\\
       \affaddr{Street Address}\\
       \affaddr{City, Country}\\
       \email{email@email.com}
\and  % use '\and' if you need 'another row' of author names
\alignauthor Author Author\\ %\titlenote{Some note on the bottom of the page.}
       \affaddr{Some Institute}\\
       \affaddr{Street Address}\\
       \affaddr{City, Country}\\
       \email{email@email.com}
\alignauthor Author Author\\ %\titlenote{Some note on the bottom of the page.}
       \affaddr{Some Institute}\\
       \affaddr{Street Address}\\
       \affaddr{City, Country}\\
       \email{email@email.com}
\alignauthor Author Author\\ %\titlenote{Some note on the bottom of the page.}
       \affaddr{Some Institute}\\
       \affaddr{Street Address}\\
       \affaddr{City, Country}\\
       \email{email@email.com}
}
% There's nothing stopping you putting the seventh, eighth, etc.
% author on the opening page (as the 'third row') but we ask,
% for aesthetic reasons that you place these 'additional authors'
% in the \additional authors block, viz.
\additionalauthors{
    Additional authors:
    Author Author (Institute, email:
    {\texttt{email@email.com}}) and
    Author Author (Institute, email:
    {\texttt{email@email.com}}).
}
\date{30 July 1999}
% Just remember to make sure that the TOTAL number of authors
% is the number that will appear on the first page PLUS the
% number that will appear in the \additionalauthors section.

\maketitle
\begin{abstract}
Abstract text.
\end{abstract}


%
% The code below should be generated by the tool at
% http://dl.acm.org/ccs.cfm
% Please copy and paste the code instead of the example below.
%
\begin{CCSXML}
<ccs2012>
 <concept>
  <concept_id>10010520.10010553.10010562</concept_id>
  <concept_desc>Computer systems organization~Embedded systems</concept_desc>
  <concept_significance>500</concept_significance>
 </concept>
 <concept>
  <concept_id>10010520.10010575.10010755</concept_id>
  <concept_desc>Computer systems organization~Redundancy</concept_desc>
  <concept_significance>300</concept_significance>
 </concept>
 <concept>
  <concept_id>10010520.10010553.10010554</concept_id>
  <concept_desc>Computer systems organization~Robotics</concept_desc>
  <concept_significance>100</concept_significance>
 </concept>
 <concept>
  <concept_id>10003033.10003083.10003095</concept_id>
  <concept_desc>Networks~Network reliability</concept_desc>
  <concept_significance>100</concept_significance>
 </concept>
</ccs2012>
\end{CCSXML}

\ccsdesc[500]{Computer systems organization~Embedded systems}
\ccsdesc[300]{Computer systems organization~Redundancy}
\ccsdesc{Computer systems organization~Robotics}
\ccsdesc[100]{Networks~Network reliability}


%
% End generated code
%

%
%  Use this command to print the description
%
\printccsdesc

% We no longer use \terms command
%\terms{Theory}

\keywords{ACM proceedings; \LaTeX; text tagging}

\section{Introduction}

Some text. \cite{example}

Exempli gratia: \eg

Id est: \ie

Et alii: \etal

Et cetera: \etc

Versus: \vs

\section{Some Section}

\subsubsection{Some Subsection}

Text.

\subsubsection{Another Subsection}

\begin{table}[!t]
    \centering
    \caption{A Table.}
    \label{tab:atable}
    \begin{tabular}{ccl}
        \toprule
        Non-English or Math & Frequency     & Comments \\
        \midrule
        \O                  & 1 in 1,000    & For Swedish names \\
        $\pi$               & 1 in 5        & Common in math \\
        \$                  & 4 in 5        & Used in business \\
        $\Psi^2_1$          & 1 in 40,000   & Unexplained usage \\
        \bottomrule
    \end{tabular}
\end{table}

\begin{figure}[!t]
    \centering
    \includegraphics[width=\columnwidth]{school-logo}
    \caption{Sample figure.}
    \label{fig:samplefig}
\end{figure}

\begin{figure}[!t]
    \centering
    \subfloat[Subfigure 1.]{
        \label{fig:subfig1}
        \includegraphics[width=0.48\columnwidth]{school-logo}
    }
    \hspace*{\fill}
    \subfloat[Subfigure 2.]{
        \label{fig:subfig2}
        \includegraphics[width=0.48\columnwidth]{school-logo}
    }
    \caption{Figure with subfigures. }
    \label{fig:subfig}
\end{figure}

Some more text. Table~\ref{tab:atable}. Figure~\ref{fig:samplefig}. Figure~\ref{fig:subfig2}.

\section{Conclusions}
Conclusions.

\section{Acknowledgments}
Thanks.

\bibliographystyle{abbrv}
\bibliography{../../references}

\balancecolumns

\end{document}
